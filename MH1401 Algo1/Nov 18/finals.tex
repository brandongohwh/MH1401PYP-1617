\documentclass[12pt]{article}
\usepackage{fancyhdr}
\usepackage{amsmath,amsfonts,enumerate,amssymb}
\usepackage{color,graphicx}
\usepackage{float}
\pagestyle{fancy}
\usepackage{soul} %Strikeout
\usepackage{minted}
\usepackage{tikzsymbols}
\usetikzlibrary{arrows}
\newcommand*\circled[1]{\tikz[baseline=(char.base)]{
		\node[shape=circle,draw,inner sep=2pt] (char) {#1};}}
%%%%%%%%%%%%%%%%%%%%%%%%%%%%%%%%%%%%%%%%%%%%%%%%%
% Do your customization here
%%%%%%%%%%%%%%%%%%%%%%%%%%%%%%%%%%%%%%%%%%%%%%%%%
\newcommand{\masunitnumber}{MH1401}
\newcommand{\examdate}{Nov 2018}
\newcommand{\academicyear}{2018-2019}
\newcommand{\semester}{I}
\newcommand{\coursename}{Algorithms and Computing I}
\newcommand{\numberofhours}{2}

\newcommand{\ZZ}{\mathbb{Z}}
\newcommand{\CC}{\mathbb{C}}
\newcommand{\RR}{\mathbb{R}}
\newcommand{\FF}{\mathbb{F}}
\newcommand{\EOQ}{\hfill $\square$}
%\DeclareMathOperator{\diam}{diam}
%%%%%%%%%%%%%%%%%%%%%%%%%%%%%%%%%%%%%%%%%%%%%%%%%
% Don't touch anything from here till instructions
% to candidates
%%%%%%%%%%%%%%%%%%%%%%%%%%%%%%%%%%%%%%%%%%%%%%%%%
\lhead{}
\rhead{}
\chead{{\bf NANYANG TECHNOLOGICAL UNIVERSITY}}
\lfoot{}
\rfoot{}
\cfoot{}
\begin{document}
\setlength{\headsep}{5truemm}
\setlength{\headheight}{14.5truemm}
\setlength{\voffset}{-0.45truein}
\renewcommand{\headrulewidth}{0.0pt}
\begin{center}
SEMESTER \semester\ EXAMINATION \academicyear ~SUGGESTED SOLUTION
\end{center}
\begin{center}
{\bf \masunitnumber\ -- \coursename}
\end{center}
\vspace{20truemm}

\noindent \examdate\hspace{55truemm} TIME ALLOWED: \numberofhours\ HOURS

\vspace{19truemm}
\hrule
\vspace{19truemm}
\noindent\underline{INSTRUCTIONS TO CANDIDATES}
\vspace{8truemm}
%%%%%%%%%%%%%%%%%%%%%%%%%%%%%%%%%%%%%%%%%%%%%%%%%%%%%%
% Adjust your instructions here
%%%%%%%%%%%%%%%%%%%%%%%%%%%%%%%%%%%%%%%%%%%%%%%%%%%%%%
\begin{enumerate}
\item This examination paper contains {\bf FOUR (4)} questions and comprises 
{\bf FIVE (5)} printed pages.

\item Answer {\bf ALL} questions. 
The marks for each question are indicated at the beginning of each question.


\item Answer each question beginning on a {\bf FRESH} page of the answer book.

\item This is a {\bf RESTRICTED OPEN BOOK} exam. Each candidate is allowed to bring {\bf ONE (1)} hand-written, double-sided A4 size help sheet.

\item Candidates may use calculators. However, they should lay out systematically the various steps in the workings.

\end{enumerate}

%%%%%%%%%%%%%%%%%%%%%%%%%%%%%%%%%%%%%%%%%%%%%%%%%
% leave this as it is
%%%%%%%%%%%%%%%%%%%%%%%%%%%%%%%%%%%%%%%%%%%%%%%%%
\newpage
\lhead{}
\rhead{\masunitnumber}
\chead{}
\lfoot{}
\cfoot{\thepage}
\rfoot{}
\setlength{\footskip}{45pt}
%%%%%%%%%%%%%%%%%%%%%%%%%%%%%%%%%%%%%%%%%%%%%%%%%%
% put your exam questions here
%%%%%%%%%%%%%%%%%%%%%%%%%%%%%%%%%%%%%%%%%%%%%%%%%%
\noindent Solutions provided by:
Brandon Goh -- bgoh008@e.ntu.edu.sg

\paragraph{Question 1.}\hfill {\bf (25 marks)}
\begin{enumerate}[(a)]
\item What is the value of the following expression?
\begin{verbatim}
10+10//2**2+5
\end{verbatim}
\item What is printed on the screen when you execute the following commands?
\begin{enumerate}[(i)]
\item \begin{verbatim}
x = 100
y = 99
z = x
x = 'hello'
y = x
print(x,y,z)
\end{verbatim}
\item \begin{verbatim}
print(list(range(3,10)))\end{verbatim}
\item \begin{verbatim}print(list(range(3,10,2)))\end{verbatim}
\item \begin{verbatim}myList = [10, 20, 'NTU', 'SPMS', 99, 999, 'hi']
ListA = myList[1:5]
ListB = myList[-1:-5:-2]
print(ListA)
print(ListB)
print(ListB*3)
\end{verbatim}
\item \begin{verbatim}
L = []
for i in range(10):
    for j in range(10):
        if (i+j)==10 and j>=i:
            L.append([i,j])
print(L)
\end{verbatim}
\end{enumerate}
\end{enumerate}
\paragraph{Answer}
\begin{enumerate}[(a)]
\item 10 (\verb|**| has highest order of precedence, followed by \verb|//| and lastly \verb|+|)
\item \begin{enumerate}[(i)]

\item `hello', `hello', 100
\item $[3,4,5,6,7,8,9]$
\item $[3,5,7,9]$
\item \text{[}20, `NTU', `SPMS', 99\text{]}\\\text{[}`hi', 99\text{]}\\\text{[}`hi', 99, `hi', 99, `hi', 99\text{]}
\item \text{[}\text{[}1, 9\text{]}, \text{[}2, 8\text{]}, \text{[}3, 7\text{]}, \text{[}4, 6\text{]}, \text{[}5, 5\text{]}\text{]}
\end{enumerate}
\end{enumerate}
\paragraph{Question 2.} \hfill {\bf (25 marks)}
\begin{enumerate}[(a)]
\item Consider the following piece of code:
\begin{verbatim}
if (age>=12):
    print('You are eligible to see the match.')
    if (age <=20 or age >=60):
        print('Ticket price is $10.')
    else:
        print('Ticket price is $15.')
else:
    print('You are not eligible.')
\end{verbatim}
Rewrite the above program so that it does the same thing without using nested \verb|if-else| statements.
\item Write a program that does the following:
\begin{itemize}
\item First, prompt the user for an input of a positive integer which will be assigned to the variable \verb|my_num|. If a user inputs $0$ or a negative integer, print an error message, and exit the program.
\item With a valid input, generate a list whose length is equal to \verb|my_num|, where each item in the list is a random integer between 1 (included) and 9 (included).
\item Then replace every odd integer in the list by 0, and print the resulting list.
\end{itemize}
\end{enumerate}
\paragraph{Answer}
\begin{enumerate}[(a)]
\item \begin{minted}[frame=single, framesep=5pt]{python}
if (age<12):
    print('You are not eligible.')
elif (20<age<60):
    print('You are eligible to see the match.')
    print('Ticket price is $15.')
else:
    print('You are eligible to see the match.')
    print('Ticket price is $10.')
\end{minted}
*Note: There are other solutions to this question.
\item \begin{minted}[frame=single, framesep=5pt,breaklines]{python}
import random as rand

my_num = int(input('Input a positive integer: '))
if my_num<=0: print('You have not entered a positive integer! Exiting ...')
else:
    //Either use for loop or list comprehension
	
    //1) Using for loop
    L=[0]*my_num
    for i in range(my_num):
        L[i]=rand.randint(1,9)
    
    //2) Using list comprehension
    L=[rand.randint(1,9) for i in range(my_num)]
    
    //Replace odd integers
    for i in range(len(L)):
        if L[i]%2==1:
            L[i]=0
print(L)
\end{minted}
\end{enumerate}\clearpage
\paragraph{Question 3.}\hfill {\bf (10 marks)}
\\\begin{verbatim}
n = int(input('Enter an integer: '))
num = 0
while (10<=n<=99):
    if n>50:
        break
    num += n
    n = int(input('Enter an integer: '))
else:
    print('ok')
print(num)
\end{verbatim}
When the program is run, what will be displayed in the output if the following numbers are entered in the given order at the prompts:
\begin{enumerate}[(i)]
\item 20, 30, 60
\item 20, 30, 40, 7
\item In general, describe the number printed on the last line of the output (i.e. how it is related to the numbers entered by the user) when the program terminates.
\item Modify the program so that it continues to prompt the user for an input of an integer so long as the previous input is a 2-digit number, i.e. it will terminate only when the integer entered is less than 10 or more than 99. Moreover, the number printed on the last line of the output should be the sum of all the \textbf{even} 2-digit integers entered by the user.
\end{enumerate}
\paragraph{Answer}
\begin{enumerate}[(i)]
\item 50
\item 90
\item The number printed on the last line is the sum of the previously entered numbers, excluding the last number (since the previous entered numbers satisfy the \textit{while} condition and \textbf{NOT} the nested \textit{if} condition)
\item \begin{minted}[frame=single, framesep=5pt,breaklines]{python}
n = int(input('Enter an integer: '))
num = 0
while (10<=n<=99):
    if n%2==0:
        num += n
    n = int(input('Enter an integer: '))
else:
    print('ok')
print(num)
\end{minted}
\end{enumerate}
\paragraph{Question 4. }\hfill {\bf (25 marks)}\\\\
Assume that the numpy module has already been imported using
\begin{verbatim}
import numpy as np
\end{verbatim}
\noindent A \textbf{Latin square} is an $n\times n$ array filled with $n$ different integers $x_1,x_2,\cdots,x_n$ such that each of these integers occurs exactly once in each row and exactly once in each column. The $(i,j)$-entry of a Latin square $L$, denoted by $L[i,j]$, is the integer that occurs in the cell at row-index $i$ and column-index $j$, assuming index starts at 0. An example of a $4\times 4$ Latin square $L$ is:
{\par \centering
\[
\begin{pmatrix}
1&6&3&7\\7&1&6&3\\3&7&1&6\\6&3&7&1
\end{pmatrix}
\]
}
where $L[0,3]=7$, $L[1,2]=6$ etc.
\begin{enumerate}[(i)]
\item Write a Python function \verb|is_latin_square| that takes a numpy array as input, and returns the boolean value \textit{True} if it is a Latin square, \textit{False} otherwise. In the case that the input numpy array does not correspond to an $n\times n$ array, the function should return the boolean value \textit{False}.
\item An \textbf{intercalate} in a Latin square $L$, denoted by $[i,j,x,y]$, is a pair of rows $i$ and $j$ and a pair of columns $x$ and $y$ such that $L[i,x]=L[j,y]$ and $L[i,y]=L[j,x]$, where $i<j$ and $x<y$. For example, the $4\times 4$ Latin square $L$ above contains an intercalate given by $[0,2,1,3]$ since $L[0,1]=L[2,3]=6$ and $L[0,3]=L[2,1]=7$.\\\\Write a Python function \verb|intercalate| that takes an $n\times n$ Latin square (represented as a numpy array) as input, and returns the following:
\begin{enumerate}[(i)]
\item the number of intercalates it contains, and
\item a list consisting of all the intercalates represented by $[i,j,x,y]$.
\end{enumerate}
\end{enumerate}
\paragraph{Answer}
\begin{enumerate}[(i)]
\item \begin{minted}[frame=single, framesep=5pt,breaklines]{python}
def my_average(mat):
    (rows,cols) = mat.shape
    my_sum = 0
    for r in range(rows):
        for c in range(cols):
            my_sum += mat[r,c]
    return my_sum/(rows*cols)
\end{minted}
\item \begin{minted}[frame=single, framesep=5pt,breaklines]{python}
def is_average(mat):
    (rows,cols) = mat.shape
    mat_average = my_average(mat)
    for r in range(rows):
        for c in range(cols):
            if mat[r,c]==mat_average:
                return True
    return False
\end{minted}
\item \begin{minted}[frame=single, framesep=5pt,breaklines]{python}
def sort_average(mat):
    (rows,cols) = mat.shape
    mat_average = my_average(mat)
    
    my_input_list = []
    for r in range(rows):
        for c in range(cols):
            my_input_list.append(mat[r,c])
    my_output_list = [my_input_list[0]]
    for j in range(1,len(my_input_list)):
        i=0
        while abs(my_output_list[i]-mat_average) < abs(my_input_list[j]-mat_average):
            i += 1
            if i==len(my_output_list):
                break
        my_output_list.insert(i,my_input_list[j])
    return my_output_list
\end{minted}
\end{enumerate}
\bigskip
\vfill
\begin{center}{\bf END OF PAPER}\end{center}
\end{document}
