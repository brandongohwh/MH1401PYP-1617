%\documentclass[12pt,notitlepage]{article}
\documentclass[a4paper,12pt]{article}
\usepackage[utf8]{inputenc}
\usepackage{graphicx}
\usepackage{verbatim}
\usepackage{amsthm}
\usepackage{pdfpages}
\usepackage{amsmath}
\usepackage{enumerate} 

\usepackage{mathtools}
\DeclarePairedDelimiter\ceil{\lceil}{\rceil}
\DeclarePairedDelimiter\floor{\lfloor}{\rfloor}

\usepackage{hyperref}
%\usepackage[T1]{fontenc}
\usepackage{url}
\usepackage{lipsum}
\usepackage{array}
\usepackage{multirow}
\usepackage{float}
\usepackage{lscape}
\usepackage{colortbl}
\newcolumntype{P}[1]{>{\centering\arraybackslash}p{#1}}
\usepackage[nottoc,numbib]{tocbibind}
\usepackage{fancyhdr}
\usepackage{hhline}
\usepackage[printonlyused]{acronym}

\usepackage{tabularx}
\usepackage{tabulary}

%\usepackage{txfonts}
\usepackage{lipsum,etoolbox}% http://ctan.org/pkg/{lipsum,etoolbox}
\usepackage{caption}
\usepackage{subcaption}
\usepackage[normalem]{ulem}

\usepackage{algorithm}
\usepackage[noend]{algpseudocode}

\makeatletter
\def\BState{\State\hskip-\ALG@thistlm}
\makeatother

\usepackage{minted}

\definecolor{black}{RGB}{0,0,0}

\usepackage{fancyvrb}

\usepackage{geometry}
\geometry{
	a4paper,
	total={170mm,257mm},
	right=3cm,
	left=3.5cm,
	top=3cm,
	bottom=3cm
}


\usepackage{titlesec}
\usepackage{hyperref}
\titleclass{\subsubsubsection}{straight}[\subsection]

\newcounter{subsubsubsection}[subsubsection]
\renewcommand\thesubsubsubsection{\thesubsubsection.\arabic{subsubsubsection}}
\renewcommand\theparagraph{\thesubsubsubsection.\arabic{paragraph}} % optional; useful if paragraphs are to be numbered

\titleformat{\subsubsubsection}
{\normalfont\normalsize\bfseries}{\thesubsubsubsection}{1em}{}
\titlespacing*{\subsubsubsection}
{0pt}{3.25ex plus 1ex minus .2ex}{1.5ex plus .2ex}

\makeatletter
\renewcommand\paragraph{\@startsection{paragraph}{5}{\z@}%
	{3.25ex \@plus1ex \@minus.2ex}%
	{-1em}%
	{\normalfont\normalsize\bfseries}}
\renewcommand\subparagraph{\@startsection{subparagraph}{6}{\parindent}%
	{3.25ex \@plus1ex \@minus .2ex}%
	{-1em}%
	{\normalfont\normalsize\bfseries}}
\def\toclevel@subsubsubsection{4}
\def\toclevel@paragraph{5}
\def\toclevel@paragraph{6}
\def\l@subsubsubsection{\@dottedtocline{4}{7em}{4em}}
\def\l@paragraph{\@dottedtocline{5}{10em}{5em}}
\def\l@subparagraph{\@dottedtocline{6}{14em}{6em}}
\makeatother

\setlength\parindent{0pt}

\setcounter{secnumdepth}{4}
\setcounter{tocdepth}{4}

\newtheorem{dummy}{Dummy} %dummy for numbering purposes

\theoremstyle{definition}
\newtheorem{definition}[dummy]{Definition}
\newtheorem{example}[dummy]{Example}
\newtheorem{exercise}[dummy]{Exercise}
\newtheorem{remark}[dummy]{Remark}
\newtheorem{fact}[dummy]{Fact}
\newtheorem{ques}[dummy]{QUESTION}

\theoremstyle{plain}
\newtheorem{theorem}[dummy]{Theorem}
\newtheorem{proposition}[dummy]{Proposition}
\newtheorem{corollary}[dummy]{Corollary}
\newtheorem{lemma}[dummy]{Lemma}


\newcommand{\ttx}[1]{\texttt{#1}}

\newcommand{\biganglebracket}[1]{\left\langle #1 \right\rangle}
\newcommand{\bigbracket}[1]{\left( #1 \right)}
\newcommand{\bigsquarebracket}[1]{\left[ #1 \right]}
\newcommand{\bigcurlybracket}[1]{\left\{ #1 \right\}}
\newcommand{\bigfloorbracket}[1]{\left\lfloor #1 \right\rfloor}
\newcommand{\bigceilbracket}[1]{\left\lceil #1 \right\rceil}
\newcommand{\bigabs}[1]{\left| #1 \right|}
\newcommand{\bignorm}[1]{\left\| #1 \right\|}

\newcommand{\py}{python}

\begin{document}
	\begin{titlepage}
		\begin{center}
			\textbf{NANYANG TECHNOLOGICAL UNIVERSITY}\\\vspace{1em}Suggested Solutions (Camille)\\\vspace{1em}
			\textbf{MH1401/CY1401 - Algorithms and Computing I}
		\end{center}
		\vspace*{4.5em}
		\noindent November 2016 \hfill TIME ALLOWED: 120 MINUTES
		\vspace*{2em}
		
		\vbox{\hrule width\linewidth height 0.5pt}
		\vspace*{3em}
		
		\noindent\underline{INSTRUCTIONS TO CANDIDATES}
		\vspace*{1.5em}
		\begin{enumerate}
			\item This examination paper contains \textbf{FOUR (4)} questions and comprises \textbf{SIX (6)} printed pages.
			
			\item Answer \textbf{all} questions. The marks for each question are indicated at the beginning of each question.
			
			\item Answer each question beginning on a \textbf{FRESH} page of the answer book.
			
			\item This \textbf{IS NOT} and \textbf{OPEN BOOK} exam.
			
			\item This paper has been converted from the original MATLAB exam to a PYTHON exam. All questions are the property of Nanyang Technological University.
		\end{enumerate}
		\vfill
	\end{titlepage}
	
	\pagenumbering{roman}
	
	\newpage
	\pagenumbering{arabic}
	
	\newpage
	\hfill MH1401/CY1401\vspace*{0.5em}
	
	
	
	\begin{ques}\hfill\textbf{(28 marks)}\\
		%		Kenneth had a little lamb\\
		%		Whose fleece was sheared by Lan\\
		%		She needed it for Bio lab\\
		%		But her experiment would not run
		\begin{enumerate}[(a)]
			\item 
			\begin{minted}{\py}
check = 0
while check == 0:
    try:
        x = int(input('Please enter a positive integer x: '))
    except ValueError:
        print('Error: Not an integer!')
    else:
        if x > 0:
            check = 1
        else:
            print('Error: Non-positive integer detected!')
			
"""
The above code implementation loops the input request
until a positive integer is entered.

If only a single input is desired, simply add in check = 1
just before the try statement.
"""
			\end{minted}
			\item False
			\item 18
			\item 
			\begin{minted}{\py}
if x < -3 or x >= 3:
    y = f(x,1)
else:
    if x > 0:
        y = f(x,2)
    else:
        if x < 0:
            y = f(x,3)
        else:
            y = f(x,4)
			\end{minted}
		\end{enumerate}
	\end{ques}
	
	\newpage
	\hfill MH1401/CY1401\vspace*{0.5em}
	
	
	\begin{ques}\hfill\textbf{(24 marks)}\\
		%		Kenneth had a little lamb\\
		%		Whose fleece was sheared by Lan\\
		%		She needed it for Bio lab\\
		%		But her experiment would not run
		\begin{enumerate}[(i)]
			\item 
			\begin{minted}{\py}
def income_tax(n):

    while n < 0:
        print('Error: income must be non-negative.')
        n = int(input('Enter income: '))
    
    if n <= 20000:
        return 0
    elif n <= 40000:
        return 0.05*n
    elif n <= 100000:
        return 0.1*n
    elif n <= 200000:
        return 0.15*n
    else:
        return 0.2*n
			\end{minted}
			\item
			\begin{minted}{\py}
def income_tax_sg(n):

    while n < 0:
        print('Error: income must be non-negative.')
        n = int(input('Enter income: '))
    
    if n <= 20000:
        return 0
    elif n <= 40000:
        return 0 + 0.05*(n-20000)
    elif n <= 100000:
        return 0 + 0.05*(20000) + 0.1*(n-40000)
    elif n <= 200000:
        return 0 + 0.05*(20000) + 0.1*(60000) + 0.15*(n-100000)
    else:
        return 0 + 0.05*(20000) + 0.1*(60000) + 0.15*(100000) \
            + 0.2*(n-200000)

"""
Note: \ is used as the linebreak operator.
It's basically just to split up long lines,
but in the code it's treated as the same line.
"""
			\end{minted}
		\end{enumerate}
	\end{ques}
	
	\newpage
	\hfill MH1401/CY1401\vspace*{0.5em}
	
	\begin{ques}\hfill\textbf{(24 marks)}\\
		%		Kenneth had a little lamb\\
		%		Whose fleece was sheared by Lan\\
		%		She needed it for Bio lab\\
		%		But her experiment would not run
		\begin{enumerate}[(i)]
			\item 
			\begin{minted}{\py}
def newton_sqrt(X,n):
    try:
        testval = int(n)
    except ValueError:
        return -1
    else:
        if X < 0 or n <= 0 or testval != n:
            return -1
        else:
            R1 = 10
            if n == 1:
                return R1
            else:
                R = newton_sqrt(X,n-1)
                Rn = 0.5*((R**2+X)/R)
                return Rn

"""
Note: I use a try statement, but if we assume
n was an integer, this will not be necessary.
""" 
			\end{minted}
			\item
			\begin{minted}{\py}
import math

def newton_sqrt_approx(X,a):
    result = a + math.sqrt(X) + 1
    counter = 0
    while abs(result - math.sqrt(X)) > a:
        counter = counter + 1
        result = newton_sqrt(X,counter)
        
    return (counter, abs(result - math.sqrt(X)))
			\end{minted}
		\end{enumerate}
	\end{ques}
	
	\newpage
	\hfill MH1401/CY1401\vspace*{0.5em}
	
	\begin{ques}\hfill\textbf{(24 marks)}\\
		%		Kenneth had a little lamb\\
		%		Whose fleece was sheared by Lan\\
		%		She needed it for Bio lab\\
		%		But her experiment would not run
		\begin{enumerate}[(i)]
			\item 
			\begin{minted}{\py}
def check_move(x,y):
    if x == []:
        return -1
    elif y == []:
        return 0
    elif x[len(x)-1] > y[len(y)-1]:
        return -1
    else:
        return 0
			\end{minted}
			\item
			\begin{minted}{\py}
def check_victory(x):
    if x == [5,4,3,2,1]:
        return 1
    else:
        return 0
			\end{minted}
			
			\item 
			\begin{minted}{\py}
rods = [[5,4,3,2,1],[],[]]
check = check_victory(rods[2])
while check == 0:
    print('Please make the next move.')
    xind = int(input('Input the index of rod x: '))
    yind = int(input('Input the index of rod y: '))
    x = rods[xind]
    y = rods[yind]
    
    if check_move(x,y) == 0:
        y.append(x[len(x)-1])
        x.pop()
        rods[xind] = x
        rods[yind] = y
        print('Mode made. Below is the current game state.')
    else:
        print('Invalid move. Below is the current game state.')
    
    check = check_victory(rods[2])
    print(rods)
    
print('Congratulations, you win!')
			\end{minted}
		\end{enumerate}
	\end{ques}
	
	
	
	
	
	
	
	
	
	
	
	
	
	
	
	
\end{document}