%\documentclass[12pt,notitlepage]{article}
\documentclass[a4paper,12pt]{article}
\usepackage[utf8]{inputenc}
\usepackage{graphicx}
\usepackage{verbatim}
\usepackage{amsthm}
\usepackage{pdfpages}
\usepackage{amsmath}
\usepackage{enumerate} 

\usepackage{mathtools}
\DeclarePairedDelimiter\ceil{\lceil}{\rceil}
\DeclarePairedDelimiter\floor{\lfloor}{\rfloor}

\usepackage{hyperref}
%\usepackage[T1]{fontenc}
\usepackage{url}
\usepackage{lipsum}
\usepackage{array}
\usepackage{multirow}
\usepackage{float}
\usepackage{lscape}
\usepackage{colortbl}
\newcolumntype{P}[1]{>{\centering\arraybackslash}p{#1}}
\usepackage[nottoc,numbib]{tocbibind}
\usepackage{fancyhdr}
\usepackage{hhline}
\usepackage[printonlyused]{acronym}

\usepackage{tabularx}
\usepackage{tabulary}

%\usepackage{txfonts}
\usepackage{lipsum,etoolbox}% http://ctan.org/pkg/{lipsum,etoolbox}
\usepackage{caption}
\usepackage{subcaption}
\usepackage[normalem]{ulem}

\usepackage{algorithm}
\usepackage[noend]{algpseudocode}

\makeatletter
\def\BState{\State\hskip-\ALG@thistlm}
\makeatother

\usepackage{minted}

\definecolor{black}{RGB}{0,0,0}

\usepackage{fancyvrb}

\usepackage{geometry}
\geometry{
	a4paper,
	total={170mm,257mm},
	right=3cm,
	left=3.5cm,
	top=3cm,
	bottom=3cm
}


\usepackage{titlesec}
\usepackage{hyperref}
\titleclass{\subsubsubsection}{straight}[\subsection]

\newcounter{subsubsubsection}[subsubsection]
\renewcommand\thesubsubsubsection{\thesubsubsection.\arabic{subsubsubsection}}
\renewcommand\theparagraph{\thesubsubsubsection.\arabic{paragraph}} % optional; useful if paragraphs are to be numbered

\titleformat{\subsubsubsection}
{\normalfont\normalsize\bfseries}{\thesubsubsubsection}{1em}{}
\titlespacing*{\subsubsubsection}
{0pt}{3.25ex plus 1ex minus .2ex}{1.5ex plus .2ex}

\makeatletter
\renewcommand\paragraph{\@startsection{paragraph}{5}{\z@}%
	{3.25ex \@plus1ex \@minus.2ex}%
	{-1em}%
	{\normalfont\normalsize\bfseries}}
\renewcommand\subparagraph{\@startsection{subparagraph}{6}{\parindent}%
	{3.25ex \@plus1ex \@minus .2ex}%
	{-1em}%
	{\normalfont\normalsize\bfseries}}
\def\toclevel@subsubsubsection{4}
\def\toclevel@paragraph{5}
\def\toclevel@paragraph{6}
\def\l@subsubsubsection{\@dottedtocline{4}{7em}{4em}}
\def\l@paragraph{\@dottedtocline{5}{10em}{5em}}
\def\l@subparagraph{\@dottedtocline{6}{14em}{6em}}
\makeatother

\setlength\parindent{0pt}

\setcounter{secnumdepth}{4}
\setcounter{tocdepth}{4}

\newtheorem{dummy}{Dummy} %dummy for numbering purposes

\theoremstyle{definition}
\newtheorem{definition}[dummy]{Definition}
\newtheorem{example}[dummy]{Example}
\newtheorem{exercise}[dummy]{Exercise}
\newtheorem{remark}[dummy]{Remark}
\newtheorem{fact}[dummy]{Fact}
\newtheorem{ques}[dummy]{QUESTION}

\theoremstyle{plain}
\newtheorem{theorem}[dummy]{Theorem}
\newtheorem{proposition}[dummy]{Proposition}
\newtheorem{corollary}[dummy]{Corollary}
\newtheorem{lemma}[dummy]{Lemma}


\newcommand{\ttx}[1]{\texttt{#1}}

\newcommand{\biganglebracket}[1]{\left\langle #1 \right\rangle}
\newcommand{\bigbracket}[1]{\left( #1 \right)}
\newcommand{\bigsquarebracket}[1]{\left[ #1 \right]}
\newcommand{\bigcurlybracket}[1]{\left\{ #1 \right\}}
\newcommand{\bigfloorbracket}[1]{\left\lfloor #1 \right\rfloor}
\newcommand{\bigceilbracket}[1]{\left\lceil #1 \right\rceil}
\newcommand{\bigabs}[1]{\left| #1 \right|}
\newcommand{\bignorm}[1]{\left\| #1 \right\|}

\newcommand{\py}{python}

\begin{document}
	\begin{titlepage}
		\begin{center}
			\textbf{NANYANG TECHNOLOGICAL UNIVERSITY}\\\vspace{1em}Suggested Solutions\\\vspace{1em}
			\textbf{MH1401/CY1401 - Algorithms and Computing I}
		\end{center}
				\vspace*{4.5em}
\noindent November 2016 \hfill TIME ALLOWED: 120 MINUTES
\vspace*{2em}

\vbox{\hrule width\linewidth height 0.5pt}
\vspace*{3em}

\noindent\underline{INSTRUCTIONS TO CANDIDATES}
\vspace*{1.5em}
\begin{enumerate}
	\item This examination paper contains \textbf{FOUR (4)} questions and comprises \textbf{SIX (6)} printed pages.
	
	\item Answer \textbf{all} questions. The marks for each question are indicated at the beginning of each question.
	
	\item Answer each question beginning on a \textbf{FRESH} page of the answer book.
	
	\item This \textbf{IS NOT} and \textbf{OPEN BOOK} exam.
	
	\item This paper has been converted from the original MATLAB exam to a PYTHON exam. All questions are the property of Nanyang Technological University.
\end{enumerate}
	\vfill
	\end{titlepage}

\pagenumbering{roman}

	\tableofcontents
	\newpage
	\pagenumbering{arabic}
\hfill MH1401/CY1401\vspace*{0.5em}
	\addcontentsline{toc}{section}{Solutions (Brandon)}
\section*{Suggested Solutions (Brandon)}
\begin{ques}\hfill\textbf{(28 marks)}\\
	%		Kenneth had a little lamb\\
	%		Whose fleece was sheared by Lan\\
	%		She needed it for Bio lab\\
	%		But her experiment would not run
	\begin{enumerate}[(a)]
		\item 
		\begin{minted}{\py}
import math

x = -1
while(x <= 0 or x!=math.floor(x)):
    try:
        x = float(input("Input a positive integer: "))
        if (x <= 0 or x!=math.floor(x)):
            print("The input is not correct!")
    except:
        x=-1
        print("The input is not correct!")

# Alternative solution

x = -1
while(x <= 0 or x!=int(x)):
    try:
        x = float(input("Input a positive integer: "))
        if (x <= 0 or x!=math.floor(x)):
            print("The input is not correct!")
    except:
        x=-1
        print("The input is not correct!")
		\end{minted}
		\item 0
		\item 18
		\item 
\begin{minted}{\py}
if x < -3 or x >= 3:
    y = f(x,1)
else:
    if x < 0:
        y = f(x,3)
    else:
        if x == 0:
            y = f(x,4)
        else:
            y = f(x,2)
\end{minted}
	\end{enumerate}
\end{ques}

\newpage
\hfill MH1401/CY1401\vspace*{0.5em}
\begin{ques}\hfill \textbf{(24 marks)}\\
	%		Kenneth had a little lamb\\
	%		Whose fleece was sheared by Lan\\
	%		She needed it for Bio lab\\
	%		But her experiment would not run
	\begin{enumerate}[(i)]
		\item 
		\begin{minted}{\py}
def income_tax(income):
    if income <= 20000:
        return 0
    elif income <= 40000:
        return 0.05 * income
    elif income <= 100000:
        return 0.1 * income
    elif income <= 200000:
        return 0.15 * income
    else
        return 0.2 * income
		\end{minted}
		
		
		\item 
		\begin{minted}{\py}
def income_tax_sg(income):
    if income <= 20000:
        return 0
    elif income <= 40000:
        return 0.05 * (income-20000)
    elif income <= 100000:
        return 0.05 * 20000 + 0.1 * (income - 60000)
    elif income <= 200000:
        return 0.05 * 20000 + 0.1 * 60000 + 0.15 * \
        (income - 100000)
    else
        return 0.05 * 20000 + 0.1 * 60000 + 0.15 * 100000 + \ 
        0.2 * (income - 200000)

# The '\' character splits code into multiple lines (making it 
# more readable)
		\end{minted}
	\end{enumerate}
\end{ques}


\newpage
\hfill MH1401/CY1401\vspace*{0.5em}
\begin{ques}\hfill \textbf{(24 marks)}\vspace*{1em}
	%		Kenneth had a little lamb\\
	%		Whose fleece was sheared by Lan\\
	%		She needed it for Bio lab\\
	%		But her experiment would not run
	\begin{enumerate}[(i)]
		\item
		\begin{minted}{\py}
def newton_sqrt(X,n):
    if X < 0 or n <= 0:
        return -1;
    if n == 1:
        R1 = 10;
        return R1;
    else:
        out = newton_sqrt(X,n-1)
        Rx = 0.5 * (out + X) / out
        return Rx;		

# Note that if they did not ask for recursion,
# we can use the following for loop instead:

# Non-recursion method:
def newton_sqrt(X,n):
    R1 = 10
    for i in range(1,n+1):
        if i-1 == 0:
            Rx = R1
        else:
            Rx = 0.5 * (R1 + x) / R1
            R1 = Rx
    return Rx
\end{minted}
\item
		\begin{minted}{\py}
def newton_sqrt_approx(X,a):
    import math
    i = 0;
    dist=a+1;
    while (i <= 0 or dist > a):
        i+=1;
        temp = newton_sqrt(X,i)
        if temp == -1:
            return (-1,0)
        else:
            dist = abs(math.sqrt(X)-(newton_sqrt(X,i)))
    return (i,dist)
    
"""
The above implementation works by adding these lines:

x=int(input("Input a non-negative number X: "))
a=float(input("Input a positive number a: "))
(n,dist) = newton_sqrt_approx(x,a)
if n == -1:
    print("Invalid input(s)!")
else:
    print("%d iterations required, distance = %.6f" % \ 
(n,dist))
"""
		\end{minted}
	\end{enumerate}
\end{ques}



\newpage

\hfill MH1401/CY1401\vspace*{0.5em}

\begin{ques}\hfill \textbf{(24 marks)}\vspace*{1em}
	%		Kenneth had a little lamb\\
	%		Whose fleece was sheared by Lan\\
	%		She needed it for Bio lab\\
	%		But her experiment would not run
	\begin{enumerate}[(i)]
\item
\begin{minted}{\py}
def check_move(x,y):
    if x == []:
        return -1
    elif y== []:
        return 0
    elif x[-1] > y[-1]:
        return -1
    else:
        return 0
\end{minted}
\item 
\begin{minted}{\py}
def check_victory(x):
    if len(x) != 5:
        return -1
    else:
        for i in range(len(x)-1):
            if x[i] <= x[i+1]:
                return -1
    return 0
\end{minted}
\newpage
\item 
\begin{minted}{\py} 
vectorofrods = [[5,4,3,2,1],[],[]]
win = -1;
while(win != 0):
    movevalid = -1
    while (movevalid != 0):
        movefrom = int(input( \
        "Which rod do you want to move the disc from? "))
        moveto = int(input( \
        "Which rod do you want to move the disc to? "))
        movevalid = check_move(vectorofrods[movefrom],\
        vectorofrods[moveto])
        if (movevalid != 0):
            print("Invalid move, try again!\n")
            print("Current rod list:", vectorofrods)
        else:
            vectorofrods[moveto].append( \
            vectorofrods[movefrom].pop())
            print("Current rod list:", vectorofrods)
    win = check_victory(vectorofrods[moveto])
print("Congratulations to you for finishing MH1401!")

"""
Part (i), (ii), (iii) is a working implementation of 
Tower of Hanoi. You can play it by copying the code 
into Spyder and running it!

Good luck! :)
"""
\end{minted}
	\end{enumerate}
\end{ques}

	\newpage
	\addcontentsline{toc}{section}{Solutions (Camille)}
	\section*{Suggested Solutions (Camille)}
\end{document}