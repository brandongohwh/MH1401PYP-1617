%\documentclass[12pt,notitlepage]{article}
\documentclass[a4paper,12pt]{article}
\usepackage[utf8]{inputenc}
\usepackage{graphicx}
\usepackage{verbatim}
\usepackage{amsthm}
\usepackage{pdfpages}
\usepackage{amsmath}
\usepackage{enumerate} 

\usepackage{mathtools}
\DeclarePairedDelimiter\ceil{\lceil}{\rceil}
\DeclarePairedDelimiter\floor{\lfloor}{\rfloor}

\usepackage{hyperref}
%\usepackage[T1]{fontenc}
\usepackage{url}
\usepackage{lipsum}
\usepackage{array}
\usepackage{multirow}
\usepackage{float}
\usepackage{lscape}
\usepackage{colortbl}
\newcolumntype{P}[1]{>{\centering\arraybackslash}p{#1}}
\usepackage[nottoc,numbib]{tocbibind}
\usepackage{fancyhdr}
\usepackage{hhline}
\usepackage[printonlyused]{acronym}

\usepackage{tabularx}
\usepackage{tabulary}

%\usepackage{txfonts}
\usepackage{lipsum,etoolbox}% http://ctan.org/pkg/{lipsum,etoolbox}
\usepackage{caption}
\usepackage{subcaption}
\usepackage[normalem]{ulem}

\usepackage{algorithm}
\usepackage[noend]{algpseudocode}

\makeatletter
\def\BState{\State\hskip-\ALG@thistlm}
\makeatother

\usepackage{minted}

\definecolor{black}{RGB}{0,0,0}

\usepackage{fancyvrb}

\usepackage{geometry}
\geometry{
	a4paper,
	total={170mm,257mm},
	right=3cm,
	left=3.5cm,
	top=3cm,
	bottom=3cm
}


\usepackage{titlesec}
\usepackage{hyperref}
\titleclass{\subsubsubsection}{straight}[\subsection]

\newcounter{subsubsubsection}[subsubsection]
\renewcommand\thesubsubsubsection{\thesubsubsection.\arabic{subsubsubsection}}
\renewcommand\theparagraph{\thesubsubsubsection.\arabic{paragraph}} % optional; useful if paragraphs are to be numbered

\titleformat{\subsubsubsection}
{\normalfont\normalsize\bfseries}{\thesubsubsubsection}{1em}{}
\titlespacing*{\subsubsubsection}
{0pt}{3.25ex plus 1ex minus .2ex}{1.5ex plus .2ex}

\makeatletter
\renewcommand\paragraph{\@startsection{paragraph}{5}{\z@}%
	{3.25ex \@plus1ex \@minus.2ex}%
	{-1em}%
	{\normalfont\normalsize\bfseries}}
\renewcommand\subparagraph{\@startsection{subparagraph}{6}{\parindent}%
	{3.25ex \@plus1ex \@minus .2ex}%
	{-1em}%
	{\normalfont\normalsize\bfseries}}
\def\toclevel@subsubsubsection{4}
\def\toclevel@paragraph{5}
\def\toclevel@paragraph{6}
\def\l@subsubsubsection{\@dottedtocline{4}{7em}{4em}}
\def\l@paragraph{\@dottedtocline{5}{10em}{5em}}
\def\l@subparagraph{\@dottedtocline{6}{14em}{6em}}
\makeatother

\setlength\parindent{0pt}

\setcounter{secnumdepth}{4}
\setcounter{tocdepth}{4}

\newtheorem{dummy}{Dummy} %dummy for numbering purposes

\theoremstyle{definition}
\newtheorem{definition}[dummy]{Definition}
\newtheorem{example}[dummy]{Example}
\newtheorem{exercise}[dummy]{Exercise}
\newtheorem{remark}[dummy]{Remark}
\newtheorem{fact}[dummy]{Fact}
\newtheorem{ques}[dummy]{QUESTION}

\theoremstyle{plain}
\newtheorem{theorem}[dummy]{Theorem}
\newtheorem{proposition}[dummy]{Proposition}
\newtheorem{corollary}[dummy]{Corollary}
\newtheorem{lemma}[dummy]{Lemma}


\newcommand{\ttx}[1]{\texttt{#1}}

\newcommand{\biganglebracket}[1]{\left\langle #1 \right\rangle}
\newcommand{\bigbracket}[1]{\left( #1 \right)}
\newcommand{\bigsquarebracket}[1]{\left[ #1 \right]}
\newcommand{\bigcurlybracket}[1]{\left\{ #1 \right\}}
\newcommand{\bigfloorbracket}[1]{\left\lfloor #1 \right\rfloor}
\newcommand{\bigceilbracket}[1]{\left\lceil #1 \right\rceil}
\newcommand{\bigabs}[1]{\left| #1 \right|}
\newcommand{\bignorm}[1]{\left\| #1 \right\|}

\begin{document}
	\begin{titlepage}
		\begin{center}
			\textbf{NANYANG TECHNOLOGICAL UNIVERSITY}\\\vspace{1em}\sout{SEMESTER 1 EXAMINATION 2016-2017} Suggested Solutions\\\vspace{1em}
			\textbf{MH1401/CY1401 - Algorithms and Computing I}
		\end{center}
				\vspace{5em}
	\underline{NOTE:}
	\begin{enumerate}
		\item The following paper has been converted from MATLAB to Python.
	\end{enumerate}
	\vfill
	\end{titlepage}

\pagenumbering{roman}

	\tableofcontents
	\newpage
	\pagenumbering{arabic}
	\addcontentsline{toc}{section}{Questions}

\begin{ques}\hfill (10 marks)\\
	%		Kenneth had a little lamb\\
	%		Whose fleece was sheared by Lan\\
	%		She needed it for Bio lab\\
	%		But her experiment would not run
	\begin{enumerate}[(a)]
		\item How do you check that an input \texttt{x} given by a user is a positive integer?
		\item Given vec=[1 0 3 0], what you be the result of the following command:\\\texttt{any(vec) \&\& all(vec)}
		\item Given the matrix\begin{equation*}
		mat = \begin{pmatrix*}
		1 & 2 & 3\\4 & 5 & 6\\7 & 8 & 9
		\end{pmatrix*}
		\end{equation*}
		what is the result of the following command:\\
		\texttt{np.sum(mat[1,3]*mat[1,:]))}
		\item Rewrite the following \texttt{if-elif} statement as a \textbf{nested} \texttt{if-else} statement that accomplishes exactly the same thing. Assume that \texttt{x} is an integer variable that has been initialised and the function \texttt{f(x,d)} is defined.\begin{verbatim}if x < -3 or x >= 3:
		y=f(x,1)
		elif x > 0:
		y=f(x,2)
		elif x < 0:
		y=f(x,3)
		else:
		y=f(x,4)
		\end{verbatim}
	\end{enumerate}
\end{ques}


\begin{ques}\hfill (10 marks)\vspace*{1em}\\
	%		Kenneth had a little lamb\\
	%		Whose fleece was sheared by Lan\\
	%		She needed it for Bio lab\\
	%		But her experiment would not run
	In Singapore, personal income tax rates for resident taxpayers are progressive. This means higher income earners pay a proportionately higher tax, with the current highest personal income tax rate at 20\%.
	
	\begin{enumerate}[(i)]
		\item Imagine a very simple tax system where a citizen pays an income tax rate with the rule from the table below.
		\begin{table}[H]
			\centering
			\begin{tabular}{|c|c|}
				\hline
				\textbf{Income} (in SGD) & \textbf{Tax rate}\\
				\hline
				$0$ to $20,000$ included & 0\&\\
				$20,001$ to $40,000$ included & 5\&\\
				$40,001$ to $100,000$ included & 10\&\\
				$100,001$ to $200,000$ included & 15\&\\
				more than $200,001$ & 20\&\\
			\end{tabular}
		\end{table}
		For example, if a citizen has an income of $45,000$ SGD, he will be in the 10\% rate section, so he will pay $45,000\times 0.1 = 4,500$ SGD. Write a function \ttx{income\_tax} that will take as input the income of the citizen, and that will return the income tax amount he has to pay.
		
		
		\item In Singapore (and in many other countries), the rule is slightly more complex as the income is taxed in layers, with a higher tax rate applied to each successive layer. Using the same Table as before, a citizen will pay a 0\% tax rate for its first $20,000$ SGD, then a 5\% tax rate for the next $20,000$ SGD, then a 10\% tax rate for its next $60,000$ SGD, etc.
		
		For example, if a citizen has an income of $145,000$ SGC, the first $20,000$ SGD are taxed at a 0\% tax rate, then the nexr $20,000$ SGD are taxed at a 5\% tax rate, then the next $60,000$ SGD are taxed at a 10\% rate, and finally the remaining $45,000$ SGD are taxed at a 15\% rate. In total, he would have to pay $(20,000\times 0 + 20,000\times 0.05 + 60,000 \times 0.1 + 45,000\times 0.15) = 13750$ SGD.
		
		Write again a function \ttx{income\_tax\_sg} that will take as input the income of the citizen, and that will return the income tax amount he has to pay for this new tax system. 
	\end{enumerate}
\end{ques}



\begin{ques}\hfill (10 marks)\vspace*{1em}\\
	%		Kenneth had a little lamb\\
	%		Whose fleece was sheared by Lan\\
	%		She needed it for Bio lab\\
	%		But her experiment would not run
	Newton's method is a method for finding successively better approximations to the roots (or zeroes) of a real-valued function. It can be used to easily find a good approximation of the square root of a number $X\geq 0$. Let $R_1 > 0$ be a rather close approximation of $\sqrt{X}$, then $R_2 = \frac{1}{2}\bigbracket{\frac{R_1+X}{R_1}}$ offers an even better approximation of $\sqrt{X}$.
	
	\begin{enumerate}[(i)]
		\item Write a \textbf{recursive} function \ttx{newton\_sqrt(X,n)} that will return the $n$-th approximation of $\sqrt{X}$ using Newton's method (starting with $R_1 = 10$ as first approximation). As error check, the function returns $-1$ when $X$ is negative or when $n$ is not a positive integer.
		
		\item Assume that you have access to the function \ttx{newton\_sqrt(X,n)} described above. Write a function \ttx{newton\_sqrt\_approx(X,a)} that will output
		\begin{itemize}
			\item how many approximation steps are needed using Newton's method (starting with $R_1 = 10$ as first approximation), so that the distance between the approximation and the real $\sqrt{X}$ value is smaller of equal to $a$, and
			\item the corresponding distance value when the sufficiently close approximation is found.
		\end{itemize}
		Warning: note that the function outputs two values (by output, we mean that the function itself outputs the value, not just a printing on the screen). Hint: you can use the built-in functions \ttx{sqrt} and \ttx{absolute} in the \ttx{numpy} package in PYTHON.
	\end{enumerate}
\end{ques}




\begin{ques}\hfill (10 marks)\vspace*{1em}\\
	%		Kenneth had a little lamb\\
	%		Whose fleece was sheared by Lan\\
	%		She needed it for Bio lab\\
	%		But her experiment would not run
	The Tower of Hanoi is a well-known mathematical game. It consists of \textbf{three rods}, and a number of disks of different sizes which can slide onto any of the three rods. The puzzle starts with all the disks stacked in ascending order of size on the first rod, the smallest at the top, thus making a conical shape (see picture below). The objective of the puzzle is to move the entire stack to the third rod (only one disk can be moved at a time), obeying the following simple rules:
	\begin{itemize}
		\item Each move consists of taking the upper disk from one of the stacks and placing it on top of another stack, i.e. a disk can only be moved if it is the uppermost disk on a stack.
		
		\item No disk may be placed on top of a smaller disk.
	\end{itemize}
	
	\begin{enumerate}[(i)]
		\item Write a \textbf{recursive} function \ttx{newton\_sqrt(X,n)} that will return the $n$-th approximation of $\sqrt{X}$ using Newton's method (starting with $R_1 = 10$ as first approximation). As error check, the function returns $-1$ when $X$ is negative or when $n$ is not a positive integer.
		
		\item Assume that you have access to the function \ttx{newton\_sqrt(X,n)} described above. Write a function \ttx{newton\_sqrt\_approx(X,a)} that will output
		\begin{itemize}
			\item how many approximation steps are needed using Newton's method (starting with $R_1 = 10$ as first approximation), so that the distance between the approximation and the real $\sqrt{X}$ value is smaller of equal to $a$, and
			\item the corresponding distance value when the sufficiently close approximation is found.
		\end{itemize}
		Warning: note that the function outputs two values (by output, we mean that the function itself outputs the value, not just a printing on the screen). Hint: you can use the built-in functions \ttx{sqrt} and \ttx{absolute} in the \ttx{numpy} package in PYTHON.
	\end{enumerate}
\end{ques}


	
	\newpage
	\addcontentsline{toc}{section}{Solutions (Brandon)}
	\section*{Suggested Solutions (Brandon)}
	\newpage
	\addcontentsline{toc}{section}{Solutions (Camille)}
	\section*{Suggested Solutions (Camille)}
\end{document}