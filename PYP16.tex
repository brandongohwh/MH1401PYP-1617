%\documentclass[12pt,notitlepage]{article}
\documentclass[a4paper,12pt]{article}
\usepackage[utf8]{inputenc}
\usepackage{graphicx}
\usepackage{verbatim}
\usepackage{amsthm}
\usepackage{pdfpages}
\usepackage{amsmath}
\usepackage{enumerate} 

\usepackage{mathtools}
\DeclarePairedDelimiter\ceil{\lceil}{\rceil}
\DeclarePairedDelimiter\floor{\lfloor}{\rfloor}

\usepackage{hyperref}
%\usepackage[T1]{fontenc}
\usepackage{url}
\usepackage{lipsum}
\usepackage{array}
\usepackage{multirow}
\usepackage{float}
\usepackage{lscape}
\usepackage{colortbl}
\newcolumntype{P}[1]{>{\centering\arraybackslash}p{#1}}
\usepackage[nottoc,numbib]{tocbibind}
\usepackage{fancyhdr}
\usepackage{hhline}
\usepackage[printonlyused]{acronym}

%\usepackage{txfonts}
\usepackage{lipsum,etoolbox}% http://ctan.org/pkg/{lipsum,etoolbox}
\usepackage{caption}
\usepackage{subcaption}
\usepackage[normalem]{ulem}

\usepackage{algorithm}
\usepackage[noend]{algpseudocode}

\makeatletter
\def\BState{\State\hskip-\ALG@thistlm}
\makeatother

\usepackage{minted}

\definecolor{black}{RGB}{0,0,0}

\usepackage{fancyvrb}

\usepackage{geometry}
\geometry{
	a4paper,
	total={170mm,257mm},
	right=3cm,
	left=3.5cm,
	top=3cm,
	bottom=3cm
}


\usepackage{titlesec}
\usepackage{hyperref}
\titleclass{\subsubsubsection}{straight}[\subsection]

\newcounter{subsubsubsection}[subsubsection]
\renewcommand\thesubsubsubsection{\thesubsubsection.\arabic{subsubsubsection}}
\renewcommand\theparagraph{\thesubsubsubsection.\arabic{paragraph}} % optional; useful if paragraphs are to be numbered

\titleformat{\subsubsubsection}
{\normalfont\normalsize\bfseries}{\thesubsubsubsection}{1em}{}
\titlespacing*{\subsubsubsection}
{0pt}{3.25ex plus 1ex minus .2ex}{1.5ex plus .2ex}

\makeatletter
\renewcommand\paragraph{\@startsection{paragraph}{5}{\z@}%
	{3.25ex \@plus1ex \@minus.2ex}%
	{-1em}%
	{\normalfont\normalsize\bfseries}}
\renewcommand\subparagraph{\@startsection{subparagraph}{6}{\parindent}%
	{3.25ex \@plus1ex \@minus .2ex}%
	{-1em}%
	{\normalfont\normalsize\bfseries}}
\def\toclevel@subsubsubsection{4}
\def\toclevel@paragraph{5}
\def\toclevel@paragraph{6}
\def\l@subsubsubsection{\@dottedtocline{4}{7em}{4em}}
\def\l@paragraph{\@dottedtocline{5}{10em}{5em}}
\def\l@subparagraph{\@dottedtocline{6}{14em}{6em}}
\makeatother

\setcounter{secnumdepth}{4}
\setcounter{tocdepth}{4}

\begin{document}
	\begin{titlepage}
		\begin{center}
			\textbf{NANYANG TECHNOLOGICAL UNIVERSITY}\\\vspace{1em}\sout{SEMESTER 1 EXAMINATION 2016-2017} Suggested Solutions\\\vspace{1em}
			\textbf{MH1401/CY1401 - Algorithms and Computing I}
		\end{center}
				\vspace{5em}
	\underline{NOTE:}
	\begin{enumerate}
		\item The following paper has been converted from MATLAB to Python.
	\end{enumerate}
	\vfill
	\end{titlepage}

\pagenumbering{roman}

	\tableofcontents
	\newpage
	\pagenumbering{arabic}
	\addcontentsline{toc}{section}{Questions}
\section*{Questions}
\textbf{QUESTION 1}
	\begin{enumerate}[(a)]
		\item How do you check that an input \texttt{x} given by a user is a positive integer?
		\item Given vec=[1 0 3 0], what you be the result of the following command:\\\texttt{any(vec) \&\& all(vec)}
		\item Given the matrix\begin{equation*}
		mat = \begin{pmatrix*}
		1 & 2 & 3\\4 & 5 & 6\\7 & 8 & 9
		\end{pmatrix*}
		\end{equation*}
		what is the result of the following command:\\
		\texttt{np.sum(mat[1,3]*mat[1,:]))}
		\item Rewrite the following \texttt{if-elif} statement as a \sout{\texttt{switch}} \textbf{nested} \texttt{if-else} statement that accomplishes exactly the same thing. Assume that \texttt{x} is an integer variable that has been initialised and the function \texttt{f(x,d)} is defined.\begin{verbatim}if x < -3 or x >= 3:
	    y=f(x,1)
	elif x > 0:
	    y=f(x,2)
elif x < 0:
    y=f(x,3)
else:
    y=f(x,4)
	 \end{verbatim}
		\end{enumerate}
	\textbf{QUESTION 2}\\
	{\par \noindent In Singapore, personal income tax rates for resident taxpayers are progressive. This means higher income earners pay a proportionately higher tax, with the current highest personal income tax rate at 20\%.}
	\newpage
	\addcontentsline{toc}{section}{Solutions (Brandon)}
	\section*{Suggested Solutions (Brandon)}
	\newpage
	\addcontentsline{toc}{section}{Solutions (Camille)}
	\section*{Suggested Solutions (Camille)}
\end{document}